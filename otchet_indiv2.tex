\documentclass[a4paper]{article}
\usepackage[14pt]{extsizes} % для того чтобы задать нестандартный 14-ый размер шрифта
\usepackage[utf8]{inputenc}
\usepackage[russian]{babel}
\usepackage{setspace,amsmath}
\usepackage[left=20mm, top=15mm, right=15mm, bottom=15mm, nohead, footskip=10mm]{geometry} % настройки полей документа

\begin{document} % начало документа
	
	% НАЧАЛО ТИТУЛЬНОГО ЛИСТА
	\begin{center}
		\hfill \break
		\hfill \break
		\hfill \break
		\hfill \break
		\hfill \break
		\hfill \break
		\hfill \break
		\hfill \break
		\hfill \break
		\hfill\break
		\hfill \break
		\hfill \break
		\hfill \break
		\hfill \break
		\hfill \break
		\hfill \break
		\hfill \break
		{\LARGE Отчёт} \\
		\hfill \break
		А.А.Федяшов, Д.Е.Теплюков\\
		\hfill \break
		13.07.2017\\
		\hfill \break
		\hfill \break
	\end{center}
	
	
	\thispagestyle{empty} % выключаем отображение номера для этой страницы
	
	% КОНЕЦ ТИТУЛЬНОГО ЛИСТА
\newpage

% Вывод содержания проекта
\newpage
	{\section{Постановка задачи}
	\hfill\break
	{\bf 
	Реализовать оконное приложение - редактор таблиц, поддерживающий следующие функции:\\}
	1)открытие таблицы из бинарного файла\\
	2)сохранение таблицы в бинарный файл\\
	3)создание пустой таблицы для последующего заполнения\\
	4)редактирование данных в ячейках\\
	5)вставка пустой строки в произвольную позицию\\
	6)удаление произвольной строки\\
	{\bf Требования к реализации:}\\
	1)Реализовать свой наследник класса QAbstractTableModel для отображения данных из файла\\
	2)Переопределить метод setData для редактирования данных в файле\\
	3)Добавить методы removeRow, insertRow для добавления и удаления записей из файла\\
	4)Использовать поток QDataStream для чтения/записи данных\\
	5)Использовать таблицу для отображения структур типа:\\
	\hfill\break
	\textit{(QString name, uint course, uint group)}\\
}
        
          
\newpage
{\section{Содержание проекта}
        \hfill\break
        \hfill\break
{\bf 1)Заголовочные файлы(.h)}\\
        \hfill\break
        \textit{redactor.h}-заголовочный файл диалогового окна. Содержит конструктор окна, заголовки функций - слотов, привязанных к определенным кнопкам в окне\\        
        \hfill\break
        \textit{studenttable.h}-заголовочный файл таблицы - наследника QAbstractTableModel. Содержит заголовки переопределенных функций для модели таблицы. Имеет определение вложенной структуры Person\\
        \hfill\break
        \hfill\break
{\bf 2)Файлы реализации(.cpp)}\\
        \hfill\break
        \textit{redactor.cpp} - содержит реализацию методов основного диалогового окна\\
        \hfill\break     
        \textit{studenttable.cpp} - содержит реализацию методов модели таблицы, а также перегрузки операторов << и >> для ввода/вывода структур Person через поток QDataStream\\        
        \hfill\break
        \textit{main.cpp} - точка входа в программу, вызывает конструктор окна Redactor и осуществляет его показ мотодом show\\
        \hfill\break
        \hfill\break
{\bf 3)Файлы интерфейса(.ui)}\\ 
       \hfill\break
       \textit{redactor.ui} - интерфейс главного диалогового окна\\
       \hfill\break
       \hfill\break       
}
	
	\newpage %Аннотация
	
	\newpage %Введение
	
	\newpage %Результаты 
	{\section{Описание классов Redactor и StudentTable}
	%Класс список	
	{\LARGE \bf StudentTable}\\
	\hfill\break
	Класс StudentTable унаследован от класса QAbstractTableModel и содержит все его невиртуальные методы. В StudentTable находится вложенная структура Person(QString,uint,uint)\\
	\hfill\break
	\hfill \break
	1){\bf Конструктор, возвращает модель данных(таблицу):}\\
	     \hfill \break
	     \textit{StudentTable(QObject*parent=0)}\\
	     \hfill \break
	     \hfill \break
	2){\bf Вставка строки в таблицу:}\\
	     \hfill \break
	     \textit{bool insertRows(int position, int rows, const QModelIndex \&index)}\\
	     \hfill\break
	     Возвращает true, если вставка прошла успешно.\\
	     \hfill \break
	     \hfill \break
	3){\bf Удаление строки в таблице:}\\
         \hfill \break	
	     \textit{bool removeRows(int position, int rows, const QModelIndex \&index)}\\
	     \hfill\break
	     Возвращает true, если вставка прошла успешно.\\
	     \hfill\break
	     \hfill\break
	4) {\bf Количество строк в таблице:}\\
	     \hfill\break
	     \textit{int rowCount(const QModelIndex \&parent = QModelIndex()) const}\\
	     \hfill\break
	     Возвращает число строк – размер контейнера для хранения структур.\\
	     \hfill\break
	     \hfill\break
	5) {\bf Количество столбцов в таблице:}\\
         \hfill\break	
	     \textit{int columnCount(const QModelIndex \&parent = QModelIndex()) const}\\
	     \hfill\break
	     Возвращает число столбцов – фиксировано 3.\\
	     \hfill\break
	     \hfill\break
	6){\bf Отрисовка данных в таблице. Вызывается как автоматически моделью, так и явно:}\\
	     \hfill\break
	     \textit{QVariant data(const QModelIndex \&index, int role = Qt::DisplayRole) const}\\
	     \hfill\break
	     Возвращает шаблонный тип данных QVariant, содержащий данные из одной ячейки.\\
	     \hfill\break
	     \hfill\break
	7){\bf Запись данных в таблицу. Позволяет вручную внести данные в ячейку таблицу:}\\
	     \hfill\break
	     \textit{bool setData(const QModelIndex \&index, const QVariant \&value, int role = Qt::EditRole)}\\
	     \hfill\break
	     Возвращает true, если запись прошла успешно.\\
	     \hfill\break
	     \hfill\break
	8){\bf Отрисовка данных в заголовках столбцов. Автоматически вызывается моделью:}\\
	     \hfill\break
	     \textit{QVariant headerData(int section, Qt::Orientation orientation, int role) const}\\
	     \hfill\break
	     В текущей реализации возвращает пустое значение.\\
	     \hfill\break
	     \hfill\break
	9){\bf Получение состояния текущей ячейки в таблице. Автоматически вызывается моделью:}\\    
	     \hfill\break
	     \textit{Qt::ItemFlags flags (const QModelIndex \&index) const}\\
	     \hfill\break
	     Возвращает тип данных флаг: ItemFlags.\\
	     \hfill\break
	     \hfill\break
	     Данные класса StudentTable:\\
	     \hfill\break
	     QVector <Person> students - контейнер для хранения структур из файла.\\
	     \hfill\break
	     StudentTable содержит перегрузки вывода структур Person в поток QDataStream. Принцип сериализации структуры состоит в ее разбиении на составные типы данных.\\
	     \hfill\break
	     \hfill\break   
	В файле studenttable.h содержится структура Person, которая хранит в себе основные строки таблицы\\
	    \hfill \break
	    \textit{QString name} - имя студента\\
	    \textit{uint course} - номер курса\\        
	    \textit{uint group} - номер группы\\
        \hfill\break
        \hfill\break 
        
 {\LARGE \bf Redactor}\\
        \hfill\break
        Класс Redactor унаследован от класса QMainWindow.\\
        \hfill\break
        
    1){\bf Конструктор, создает экземпляр диалогового окна:}\\
        \hfill\break
        \textit{Redactor(QWidget *parent = 0)}\\
        \hfill\break
        Создает модель типа StudentTable, затем присваивает ее виджету для таблиц QTableWidget.\\
        \hfill\break
        \hfill\break
    2){\bf Слот открытия таблицы:}\\
        \hfill\break
        \textit{void on\_actionOpen\_triggered()}\\
        \hfill\break
        Создает диалоговое окна выбора файла, затем открывает поток по выбранному имени. Данные из файла переносятся в новую модель таблицы, которая затем присваивается виджету.\\
        \hfill\break
        \hfill\break
    3){\bf Слот сохранения таблицы:}
        \hfill\break
        \textit{void on\_actionSave\_triggered()}\\
        \hfill\break
        Создает диалоговое окно создания файла, затем открывает поток по введенному имени файла. Данные из таблицы построчно записываются в файл.\\
        \hfill\break
        \hfill\break
    4){\bf Слот создания новой таблицы:}\\
        \hfill\break
        \textit{void on\_actionCreate\_new\_triggered()}\\
        \hfill\break
        Создает новую таблицу из трех пустых строк.\\
        \hfill\break
        \hfill\break
    5){\bf Слот открытия диалогового окна About:}\\
        \hfill\break
        \textit{void on\_actionAbout\_triggered()}\\
        \hfill\break
        Создает новое окно с помощю QMessageBox::about и заполняет его данными о разработчиках.\\
        \hfill\break
        \hfill\break
    6){\bf Слот закрытия программы:}\\
        \hfill\break
        \textit{void on\_actionQuit\_triggered()}\\
        \hfill\break
        Вызывает метод QApplication::quit().\\
        \hfill\break
        \hfill\break
    7){\bf Слот кнопки для добавления новой строки в таблицу:}\\
        \hfill\break
        \textit{void on\_addButton\_clicked()}\\
        \hfill\break
        Вызывает метод StudentTable::InsertRows и получает индекс выбранной ячейки в таблице.\\
        \hfill\break
        \hfill\break
    8){\bf Слот кнопки для удаления выбранной строки в таблице:}\\
        \hfill\break
        \textit{void on\_deleteButton\_clicked()}\\
        \hfill\break
        Вызывает метод StudentTable::RemoveRows и получает индекс выбранной ячейки в таблице.\\
        \hfill\break
        \hfill\break               
    }
       
    {\section{Особенности реализации программы}
    Программа является полноценным редактором таблиц, реализующим связку модель данных <-> виджет.\\
    
    Для улучшения внешнего вида приложения была добавлена иконка, которая отображается в заголовке окна и в диалоговом окне About.\\
    
    Для невозможности растяжения окна после запуска приложения были установлены максимальные и минимальные размеры окна, с соответствующей блокировкой кнопки “На весь экран”.\\
    
    Кнопки Save, Create New, Open, About и Quit были внесены в закладку File, расположенную в верхней части окна. Для наглядности, некоторые действия были разграничены линиями.\\
    
    В программе производится обработка ошибочных ситуаций, таких как попытка удаления последней строки в файле. При ошибке открытия/сохранения файла выводится диалоговое окно QMessageBox::critical.\\      
    }   	            
\end{document}